% Based on:
% https://v1.overleaf.com/latex/templates/vcu-math-490-review-of-an-interesting-article/smwgtrztrnsn.pdf
% and https://www.overleaf.com/latex/templates/sbc-conferences-template/blbxwjwzdngr

\documentclass[12pt]{article}

\usepackage{sbc-template}

\usepackage{graphicx,url}

%\usepackage[brazil]{babel}   
\usepackage[utf8]{inputenc}  

\sloppy

\title{The Google File System Review}

\author{Thiago A. Sposito\inst{1} }


\address{Departamento de Ciência da Computacão Universidade Federal de Minas Gerais (UFMG)}

\begin{document} 

\maketitle


\section{Introduction}

The Google File System is an article written in 2003 by Sanjay Ghemawat, Howard Gobioff, and Shun-Tak Leung \cite{ghemawat2003google}. The three of them were Google's employees back in the time.

By the time of publication, google was a five years old company with around 800 permanent employees \cite{google_coorp_info} and annual revenue of around 1 billion dollars. 2003 was the year prior to the company's IPO\cite{katje_2020}. Therefore, it was in their best interest to promote how this company was growth capable without significant scale frictions. This article proposes a distributed file system designed to serve the company internally.

\section{Summary} 
The first couple sections of the paper are a general description of the problem and their take on tackling it, including their assumptions and proposed architecture. The third section is all about how the system's different components and stakeholders interact with each other. It presents the control/data flows and responsibilities. The following section is a more in-depth description of Master/Workers architectures, presenting the design choices for dealing with their forecast bottlenecks in scalability. Section 5 presents the reader on system reliability and mechanisms to enforce it. The following two sections are respectively objective and subjective observations and analyses of the system. The final sections are comparisons with related works conclusions and acknowledgements. 


\section{Analysis}




\section{Conclusion}




\bibliographystyle{sbc}
\bibliography{sbc-template}

\end{document}
