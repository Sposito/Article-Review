% Based on:
% https://v1.overleaf.com/latex/templates/vcu-math-490-review-of-an-interesting-article/smwgtrztrnsn.pdf
% and https://www.overleaf.com/latex/templates/sbc-conferences-template/blbxwjwzdngr

\documentclass[12pt]{article}

\usepackage{sbc-template}

\usepackage{graphicx,url}

%\usepackage[brazil]{babel}   
\usepackage[utf8]{inputenc}  

\sloppy

\title{Shenango: Achieving High CPU Efficiency for Latency-sensitive Datacenter Workloads Review}

\author{Thiago A. Sposito\inst{1} }


\address{Departamento de Ciência da Computação Universidade Federal de Minas Gerais (UFMG)}

\begin{document} 

\maketitle


\section{Introduction}
This paper is a tutorial that invites the reader to learn about fast packet processing, how they are presented to the extended Barkley Packer Filter, and a couple of hooks: eXpress Data Paths (XDP) and Traffic Control(TC). The paper seems friendly for the "uninitiated", covering the main concepts, architectures, and practical uses and showing a good amount of hands-on code. \cite{vieira2020fast}
\section{Analysis}
After introductions, the article presents the Original BPF from 1992. It describes its origins as a kernel packet filtering tool running bytecode transferred to kernel space. BPF stands for both the instruction set and the execution environment.
This trail paper describes the extended version of BPF, introduced on Linux kernel 2.5 in 2011. We can cite some of its new features: increase in register numbers registers width increase to 64bit and a stack of 512 bytes and support for C like function calls. eBPF uses a restricted dialect of C, with changes made mainly to allow safety verifications that must be done just before JIT compilation.

Later on, the article goes on the kernel networking layer and describe two hooks that the paper will use in the tutorial: XDP, the lowest level of the network stack in the kernel. For handling package transmission not supported by XDP, the Traffic Control Hook is presented. 

From here article gives the reader a few examples, with descriptions and diagrams. It describes a TCP filter, a user/kernel interaction example, and coordinating TC and XDP using the eBPF. 

After some more discussion about tooling and other platforms that make use of eBPF, it finalizes with the summarization of some projects that makes use of eBPF.

\section{Conclusion}
The article presents some theoretical and practical aspects of eBPF in network space, mainly related to packet processing. IT also gives the reader a good overview and understanding of how eBPF can unlock some flexibility in a needlessly static kernel area as it is the networking. I finished the article with the impression eBPF has the potential to do with networking what General Purpose GPU programming did for graphics.

\bibliographystyle{sbc}
\bibliography{sbc-template}

\end{document}
