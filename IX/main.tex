% Based on:
% https://v1.overleaf.com/latex/templates/vcu-math-490-review-of-an-interesting-article/smwgtrztrnsn.pdf
% and https://www.overleaf.com/latex/templates/sbc-conferences-template/blbxwjwzdngr

\documentclass[12pt]{article}

\usepackage{sbc-template}

\usepackage{graphicx,url}

%\usepackage[brazil]{babel}   
\usepackage[utf8]{inputenc}  

\sloppy

\title{ IX: A Protected Dataplane Operating System for High Throughput and Low Latency Review}

\author{Thiago A. Sposito\inst{1} }


\address{Departamento de Ciência da Computação Universidade Federal de Minas Gerais (UFMG)}

\begin{document} 

\maketitle

\section{Analysis}
On a conventional Linux server, it is common to use a user-level network stack in order to bypass the kernel and gain performance by context switch overheads avoidance. IX \cite{belay2014ix}proposes a solution for this problem by making a separation of the regular kernel responsible for scheduling and system control from a  virtualized network stacks called by the author of the "dataplanes". It uses the Dune \cite{belay2012dune} Linux module to be able to run those dataplanes in hardware ring 0 while maintaining full protection.

That said, raw virtualization is not enough to provide satisfactory performance boots; in order to diminish latency, the authors describe some techniques used by IX; it does pass incoming network input as pointers avoiding the neep of copies altogether; it implements a timeout to the incoming batch, not necessarily waiting it to fill before dispatching it. It also locks correlated flows in the same physical core to avoid inter-core communication overheads; finally, it does use a congestion monitor to decide when to allocate and deallocate cores for dataplanes.

\section{Conclusion}
IX proved itself a success and showed much better use of Dune than the benchmark test applications described in \cite{belay2012dune}. Using a single processor socket, it was capable of saturating 4x10GB configurations. Also, the paper authors mention that porting memcached a key-value store\cite{jose2011memcached} to IX, they were able to see more than 3x throughput and drop tail latency in half.
\bibliographystyle{sbc}
\bibliography{sbc-template}

\end{document}
